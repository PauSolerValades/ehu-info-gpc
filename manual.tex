\documentclass[12pt,a4paper]{article}
\usepackage[utf8]{inputenc}
\usepackage{amsmath}
\usepackage{amsfonts}
\usepackage{amssymb}
\usepackage{graphicx}
\usepackage{color}
%%%%%%%%%%%%%%%%%%%%%%%%%%%%%% User specified LaTeX commands.
\makeatletter\makeatother
\usepackage{listings}

% Paketea konfiguratu behar dugu C lengoaiarekin erabiltzeko:
\definecolor{darkgreen}{rgb}{0,0.5,0}
\definecolor{lightgray}{rgb}{0.95,0.95,0.95}
\definecolor{gray}{rgb}{0.65,0.65,0.65}
\lstset{language=C,
		basicstyle=\scriptsize\ttfamily,
		keywordstyle=\color{darkgreen}\bfseries,
		identifierstyle=\color{blue},
		commentstyle=\color{gray}, 
		stringstyle=\ttfamily,
		showstringspaces=false,
		tabsize=2,
		backgroundcolor=\color{lightgray}}


\begin{document}

\title{\begin{center}
\resizebox*{0.50\textwidth}{0.15 \textheight}{
\includegraphics{ehu-pdf}
} 
\end{center}
\vspace{1cm}
Ejemplo de manual}

\author{J. Makazaga}

\maketitle
\begin{abstract}
Ejemplo de documentación de una aplicación implementada en C y documentada mediante Latex.

\end{abstract}

\section{Objetivos de nuestra aplicación}

Esta maravillosa aplicaciñon tiene como objetivo\ldots

\section{Sobre este documento}

Se trata de un documento editado mediante un editor cuyo único requerimiento es que escriba simplemente texto. Dicho texto puede ser contenido del documento o comandos de latex. Los comandos indican que empieza una nueva sección, capítulo o lo que sea, o incluso alguna referencia bibliográfica o referencia interna del deldocumento\ldots Tras la edición del documento hay que procesarlo mediante un comando en la terminal:
\begin{verse}
\$ latex manual.tex
\end{verse}
Ese comando genera el docuemnto \emph{manual.dvi} y partiendo de éste último documento podemos generar el documento pdf:
\begin{verse}
\$ dvipdf manual.dvi
\end{verse}

\section{Interacción con la usuaria}
Las opciones de nuestra aplicación son: 
\begin{enumerate}
\item teclas del teclado

Sirven para interactuar con los objetos tal y como se especifica en la sección \ref{sec:ficheros-de-objetos}

\item ratón
\item mirada. 

La mirada permite un grado de interacción imposible hasta ayer. Nuestra aplicación nos da esa posibilidad.
\end{enumerate}

\section{Ficheros de objetos tridimensionales\label{sec:ficheros-de-objetos}}

Utilizamos ficheros con la definición de los objetos a visualizar. El formato de los ficheros es el establecido en el libro \cite{key-1}


\section{Código C}

Se puede utilizar el paquete \textit{listings}. En el preámbulo del documento se puede ver la configuración del paquete. Su uso es muy sencillo!

% entorno lstlisting para escribir código
\begin{lstlisting}
#include <stdio.h>
#include <malloc.h>

int main(int argc, char** argv) {
	exit(0);
}
\end{lstlisting}

\section{Trabajo futuro}

Habrá que mantener y mejorar la aplicación.

\begin{thebibliography}{Formats}
\bibitem[Fromats]{key-1}Autor, \emph{Formatos de objetos tridimensionales}, 2023 
\end{thebibliography}

\end{document}
